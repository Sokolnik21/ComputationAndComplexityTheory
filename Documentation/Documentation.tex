\documentclass[12pt, letterpaper, twoside]{article}
\usepackage[utf8]{inputenc}
\usepackage{polski}
\usepackage{hyperref}

\title{
	Algorytmy dla Problemów Trudnych Obliczeniowo \\
	\large --projekt--}
\author{Michał Sokół}
\date{lato 2019}
 
\begin{document}

\begin{titlepage}
\maketitle
\end{titlepage}

% informacje na temat użytych algorytmów i pomysłów rozwiązania zadania

\begin{abstract}
Celem tego projektu jest zaproponowanie najoptymalniejszego algorytmu dla gry "Inertia" autorstwa Simona Tathama. \href{https://www.chiark.greenend.org.uk/~sgtatham/puzzles/js/inertia.html}{link do gry}
\end{abstract}
 
\part{Wymagania wstępne}
\section{Języki programowania}
Jedyne dwa dopuszczalne języki programowania to C oraz C++/C++11.
\section{Współbieżność}
Programy nie mogą odwoływać się do systemu plików, nie mogą próbować nawiązywać połączeń sieciowych, być wielowątkowe, wieloprocesorowe, itp. Programy powinny jedynie wczytać ze standardowego wejscia opis zadania, wykonać obliczenia w pamięci i wypisać wynik na standardowe wyjście.
\newpage

\part{Implementacja}
\section{Reprezentacja danych}

\section{Algorytm \#1}
\begin{enumerate}
	\item{}
	\item{one}
\end{enumerate}

\end{document}

